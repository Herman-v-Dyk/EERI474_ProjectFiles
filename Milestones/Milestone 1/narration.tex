\documentclass[a4paper, 12pt]{report}


\begin{document}
	Good day everyone. I'm Herman van Dyk and this is my problem contextualisation presentation. My project is the medical wristband body temperature monitor.
	\\\\
	Click
	\\\\
	
	
	 Biochemical processes take place inside the living cells of the human body and are greatly influenced by body temperature. When the body temperature of an individual is too high or too low, metabolic activities can be altered, tissue damage could be caused and organic functions could be disturbed. Therefore, body temperature greatly affects one’s physical health and is a critical variable in health and disease.
	\\\\
	Click
	\\\\
	The body needs to maintain its temperature at a certain level to support its metabolic activities, human body temperature is therefore regulated at 37°C ± 1°C. 
	\\\\
	Click
	\\\\
	The measurement of body temperature plays an important role in our everyday life (especially in the pandemic we find our-self in) since several diseases are characterised by a change in body temperature. Hence, it is important to examine and monitor body temperature to hunt for signs of diseases such as covid-19 and heat stroke. 
	\\\\
	Click
	\\\\
	Doctors and other medical practitioners use body temperature measurements to analyse the effectiveness of the prescribed treatment on their patients. 
	\\\\
	Click
	\\\\
	
	Generally there are two major categories of body temperature measuring devices. Invasive type devices that can measure core/deep tissue body temperature and non-invasive devices that measures surface temperature. Invasive temperature measurements can be taken through the oral cavity, ear canal, and rectum, whilst non-invasive readings are made on
	the skin surface.
	\\\\
	Click
	\\\\
	Various types of body temperature measuring devices already exist, some of the most popular domestic-used devices are:
	\begin{itemize}
		\item Oral Thermometer:\\is an invasive device which uses thermistor resistance that varies with temperature as sensing element.
		\item Tympanic Thermometer:\\measures the natural emission of infrared thermal radiation from the tympanic membrane. Tympanic Thermometers are minimally invasive since it is placed inside the ear canal to take body temperature readings.
		\item Mercury-in-glass / Alcohol-in-glass Thermometer:\\is an invasive device that displays the thermal expansion of the ethanol/mercury caused by heat. 
		\item Infrared Thermometer:\\uses a pyroelectric sensor, to measure temperature. This is a non-invasive device that measures thermal radiation (infrared) emitted from the forehead and skin to deduce body temperature.
	\end{itemize}	
	  
	Cick
	\\\\
	
	Each category (Referring to Invasive and Non-Invasive) has their own advantages and disadvantages. Invasive devices can measure core body temperature more accurately than non-invasive devices since non-invasive devices only measures skin temperatures, which can easily be affected by external factors such as direct sunlight and indoor heating/cooling. Invasive devices may be unhygienic to use when not properly cleaned and disinfected after use. In many cases invasive methods cannot be used such as when someone is unconscious, confused, or sneezing repeatedly, then the oral measurement method is unsuitable and when someone has a middle ear infection, the ear method is of no use. Invasive devices may also be very uncomfortable for some. 
	\\\\
	Click
	\\\\
	From the previous points it is clear that a device is needed that can measure core body temperature in a non-invasive manner, without measurements getting affected, and being simple and comfortable. 
	\\\\
	Click
	\\\\
	A possible solution to the problem is to design and implement a medical wristband that can measure body temperature. This
	will allow end-users to wear the device on their wrist, and to see his or her body temperature that is measured and displayed by the wristband.
	\\\\
	Click
	\\\\
	
	The primary objective of this project is to develop an accurate body temperature measuring device that can be worn as a wristband. The device should measure one’s body temperature without the readings getting affected by external factors. The wristband must do this whilst being simple, low-cost, lightweight and energy efficient. 
	\\
	Since the device will predict core body temperature, a thermal equivalent circuit model will be used to measure core body temperature with a skin-attachable sensor. This thermal equivalent circuit must be optimised to improve the accuracy of the measurements.
	\\\\
	Click
	\\\\
	The measured body temperature must also be displayed to the user and therefore a secondary objective is to implement a Human-Machine Interface. 
	
\end{document}